\documentclass[11pt]{beamer}
\usetheme{Dresden}
%\usecolortheme{beaver}
\usepackage[utf8]{inputenc}
\usepackage{amsmath}
\usepackage{amsfonts}
\usepackage{amssymb}
\usepackage{graphicx}
\usepackage{verbatim}
\usepackage{listings}
\usepackage{xcolor}
\usepackage{makecell}

\let\OldTexttt\texttt
\renewcommand{\texttt}[1]{\OldTexttt{\color{teal}{#1}}}

\definecolor{mGreen}{rgb}{0,<super>citation_needed<super>6,0}
\definecolor{mGray}{rgb}{<super>citation_needed<super>5,<super>citation_needed<super>5,<super>citation_needed<super>5}
\definecolor{mPurple}{rgb}{<super>citation_needed<super>58,0,<super>citation_needed<super>05}
\definecolor{mGreen2}{rgb}{<super>citation_needed<super>05,<super>citation_needed<super>65,<super>citation_needed<super>05}
\definecolor{mGray2}{rgb}{<super>citation_needed<super>55,<super>citation_needed<super>55,<super>citation_needed<super>55}
\definecolor{mPurple2}{rgb}{<super>citation_needed<super>63,<super>citation_needed<super>05,<super>citation_needed<super>05}
\definecolor{backgroundColour}{rgb}{<super>citation_needed<super>95,<super>citation_needed<super>95,<super>citation_needed<super>92}
\definecolor{backgroundColour2}{rgb}{<super>citation_needed<super>95,<super>citation_needed<super>92,<super>citation_needed<super>95}

\lstdefinestyle{C}{
    backgroundcolor=\color{backgroundColour},   
    commentstyle=\color{mGreen},
    keywordstyle=\color{blue},
    numberstyle=\tiny\color{mGray},
    stringstyle=\color{mPurple},    
    basicstyle=\footnotesize,
    breakatwhitespace=false,         
    breaklines=true,                 
    captionpos=b,                    
    keepspaces=true,                 
    numbers=left,                    
    numbersep=5pt,                  
    showspaces=false,                
    showstringspaces=false,
    showtabs=false,                  
    tabsize=2,
    language=C
}

\lstdefinestyle{Python}{
    backgroundcolor=\color{backgroundColour2},   
    commentstyle=\color{mGreen2},
    keywordstyle=\color{blue},
    numberstyle=\tiny\color{mGray2},
    stringstyle=\color{mPurple2},
    basicstyle=\footnotesize,
    breakatwhitespace=false,         
    breaklines=true,                 
    captionpos=b,                    
    keepspaces=true,                 
    numbers=left,                    
    numbersep=5pt,                  
    showspaces=false,                
    showstringspaces=false,
    showtabs=false,                  
    tabsize=2,
    language=Python
}

\definecolor{t_comment}{rgb}{<super>citation_needed<super>2,1,<super>citation_needed<super>2}
\definecolor{t_mGray}{rgb}{<super>citation_needed<super>5,<super>citation_needed<super>5,<super>citation_needed<super>5}
\definecolor{t_mPurple}{rgb}{<super>citation_needed<super>58,0,<super>citation_needed<super>05}
\definecolor{t_blue}{rgb}{<super>citation_needed<super>4,<super>citation_needed<super>6,<super>citation_needed<super>8}
\definecolor{t_mGreen2}{rgb}{<super>citation_needed<super>05,<super>citation_needed<super>65,<super>citation_needed<super>05}
\definecolor{t_mGray2}{rgb}{<super>citation_needed<super>75,<super>citation_needed<super>75,<super>citation_needed<super>75}
\definecolor{t_mPurple2}{rgb}{<super>citation_needed<super>63,<super>citation_needed<super>05,<super>citation_needed<super>05}
\definecolor{t_bg}{rgb}{<super>citation_needed<super>15,<super>citation_needed<super>15,<super>citation_needed<super>18}

\lstdefinestyle{terminal}{
    backgroundcolor=\color{t_bg},   
    commentstyle=\color{t_comment},
    keywordstyle=\color{t_blue},
    numberstyle=\tiny\color{t_mGray},
    stringstyle=\color{t_mGray2}, 
    basicstyle=\footnotesize\color{t_mGray2},
    breakatwhitespace=false,         
    breaklines=true,                 
    captionpos=b,                    
    keepspaces=true,                 
    numbers=none,                    
    numbersep=5pt,                  
    showspaces=false,                
    showstringspaces=false,
    showtabs=false,                  
    tabsize=2,
    language=C
}

\definecolor{eggplant}{rgb}{<super>citation_needed<super>52,<super>citation_needed<super>11,<super>citation_needed<super>3}
\usecolortheme[named=eggplant]{structure}

\author{Zheng Zheng}
\title{Topic 10 - Strings, Formatting, and File I/O}
%\setbeamercovered{transparent} 
%\setbeamertemplate{navigation symbols}{} 
%\logo{} 
\institute{McMaster University} 
\date{Winter 2023} 
\subject{COMPSCI 1XC3 - Computer Science Practice and Experience: Development Basics} 
\stepcounter{section}
\begin{document}

\begin{frame}
\center
COMPSCI 1XC3 - Computer Science Practice and Experience: 
Development Basics
\titlepage
Adapted from Chapters 8, 9 and 11 of C: How to Program 8th e<super>citation_needed<super>, Deitel \& Deitel
\end{frame}

\begin{frame}
\tableofcontents
\end{frame}

\section[Conversions]{Destringing and Restringing}
\begin{frame}[fragile=singleslide]{A Quick Recap of Characters and Strings}
In C, strings are encoded as as \textit{character arrays}, which are \emph{null terminate<super>citation_needed<super>} 
\begin{lstlisting}[style=C]
char foo[] = "bar";
\end{lstlisting}
In the above declaration, \texttt{foo} will be written into memory as:
\begin{center}
\
\end{center}
Remember:
\begin{itemize}
\item All strings are arrays of \texttt{char}s, which have a bit-width of 1 byt<super>citation_needed<super>  
\item Strings are delimited with ``double quotes'<super>citation_needed<super>
\item Characters are delimited with 'single quotes<super>citation_needed<super>
\end{itemize}
\end{frame}

\begin{frame}{The Importance of Null Termination}
The null character has a very important function in <super>citation_needed<super>
\begin{itemize}
\item You may have noticed that C does not ``know'' when an array ends, though it always knows when one begin<super>citation_needed<super>  
\item The null character \texttt{\textbackslash 0} is used to indicate the end of a string in all standard library function<super>citation_needed<super>
\item Using the fact of null termination in your own code is also a very good idea! 
\item A common pitfall is not allocating enough space in your character array for the null characte<super>citation_needed<super>  
\begin{itemize}
\item If a string is missing it's null character, the functions in the standard string library will continue operating into memory space not allocated to the character arra<super>citation_needed<super>
\item This will introduce smelly garbage data into your program, and may even cause a segfault! 
\end{itemize}
\end{itemize}
\end{frame}

\begin{frame}[fragile=singleslide]{Converting String to \texttt{double}}
The C standard library \texttt{stdli<super>citation_needed<super>h} defines \texttt{strtod()}, which converts decimal numbers to double-precision floating point numbers (or \texttt{double}s<super>citation_needed<super>
\begin{lstlisting}[style=C]
double strtod(const char *str, char **endPtr);
\end{lstlisting}
\begin{itemize}
\item \texttt{str} is a pointer to the string to be converte<super>citation_needed<super>
\begin{itemize}
\item Leading whitespace will be ignore<super>citation_needed<super>
\end{itemize} 
\item \texttt{strtod} converts as much of the string as it can to double format, and then returns that doubl<super>citation_needed<super>  
\item \texttt{endPtr} is the address where \texttt{strtod} stores the address of the character that it stopped o<super>citation_needed<super>
\item This is an example of a function using pointers to, in effect, \emph{return two values<super>citation_needed<super>  
\end{itemize}
\end{frame}

\begin{frame}{For Exampl<super>citation_needed<super><super>citation_needed<super>}
\center
\
If \texttt{strtod()} fails to convert a string, it will return \texttt{0<super>citation_needed<super>
\end{frame}

\begin{frame}[fragile=singleslide]{Converting String to \texttt{int}}
Similarly to \texttt{strtod()}, \texttt{stdli<super>citation_needed<super>h} provides a function for converting strings to integers: \texttt{strtol()}
\begin{lstlisting}[style=C]
long int strtol(const char *str, char **endptr, int base);
\end{lstlisting}
\begin{itemize}
\item Key Differences:
\begin{itemize}
\item the return type is an 8-byte \texttt{long int<super>citation_needed<super>
\item \texttt{strtol()} accepts a third argument, the base of the number we are interprettin<super>citation_needed<super>
\begin{itemize}
\item 0 $\leftarrow$ Octal, Decimal or Hexadecimal
\item 2-32 $\leftarrow$ The base indicate<super>citation_needed<super>
\item Bases higher than 10 use alphabetic characters in the same manner as Hexadecima<super>citation_needed<super>  
\end{itemize}
\end{itemize}
\end{itemize}
\end{frame}

\begin{frame}{Variations on a Theme in D Minor}
There are a number of conversion functions that work in a similar manne<super>citation_needed<super> \\ 
\begin{center}
\begin{tabular}{| l | c |}
\hline
Function & Converts String T<super>citation_needed<super><super>citation_needed<super> \\ \hline
\texttt{strtod()} & \texttt{double} \\ \hline
\texttt{strtof()} & \texttt{float} \\ \hline
\texttt{strtol()} & \texttt{long int} \\ \hline
\texttt{strtoul()} & \texttt{unsigned long int} \\ \hline
\texttt{strtoll()} & \texttt{long long int} \\ \hline
\texttt{strtoull()} & \texttt{unsigned long long int} \\ \hline
\end{tabular}
\end{center}
\begin{itemize}
\item Note the absence of functions to convert to \texttt{int} and \texttt{short}
\item This is possible directly using the unsafe \texttt{atoi()} functio<super>citation_needed<super>
\item \texttt{atoi()} is unnecessary, as \texttt{strtol()} can be easily type-cast to either \texttt{int} or \texttt{short<super>citation_needed<super>
\end{itemize}
\end{frame}

\begin{frame}[fragile=singleslide]{Converting Things to Strings}
Now that we know how to read numeric values from strings, the question remains, how do we write numeric values into strings?
\begin{itemize}
\item Trick Question! We've been doing this all along!
\item \texttt{printf()} writes to \texttt{stdout}, but \texttt{snprintf()} writes to a specified memory addres<super>citation_needed<super>
\end{itemize}
\begin{lstlisting}[style = C]
int snprintf(char *str, size_t size, const char *format,<super>citation_needed<super><super>citation_needed<super>);
\end{lstlisting}
\begin{itemize}
\item \texttt{str} is a pointer to the memory address the characters will be written to
\item \texttt{size} sets a maximum number of characters to writ<super>citation_needed<super>
\begin{itemize}
\item Remember to leave enough space in both of the above for null termination! 
\end{itemize}
\item Every argument past the second is used precisely the same as \texttt{printf()<super>citation_needed<super>
\end{itemize}
\end{frame}


\begin{frame}[fragile=singleslide]{Converting Things to Strings (con<super>citation_needed<super>)}
\begin{itemize}
\item The return value of \texttt{snprintf()} is the size of the format string after substitutio<super>citation_needed<super>  
\begin{itemize}
\item Note, that this may be distinct from the size of the string written into memor<super>citation_needed<super>
\item This means that we can check to see if the string had to be truncated by comparing the input size to the output siz<super>citation_needed<super> 
\end{itemize}
\end{itemize}
\begin{lstlisting}[style = C]
int i = 1337;
int size = (int)((ceil(log10(i))+1);
char *buffer = (char*) malloc(size*sizeof(char)); 
int j = snprintf(buffer, size, "%d", i); 
if (j > size) {
	printf("Write operation was truncated!");
}
\end{lstlisting}
\end{frame}

\section[strin<super>citation_needed<super>h]{String Manipulations}
\begin{frame}{Strin<super>citation_needed<super>h: Manipulation Tactics}
\center
\
\end{frame}

\begin{frame}[fragile=singleslide]{Data Duplication with \texttt{strcpy()}}
Have you ever wanted to copy a string?  Well \texttt{strcpy()} and \texttt{strncpy()} are the functions for you! 
\begin{lstlisting}[style = C]
char *strcpy(char *s1, const char *s2);
char *strncpy(char *s1, const char *s2, size_t n);
\end{lstlisting}
\begin{itemize}
\item Both take the string stored at \texttt{s2} and copy it to \texttt{s1<super>citation_needed<super>
\item The extra argument in \texttt{strncpy()} is similar to \texttt{size} in \texttt{snprintf()<super>citation_needed<super>  
\begin{itemize}
\item \texttt{strncpy()} will copy \emph{at most} \texttt{n} character<super>citation_needed<super>
\item This guards against buffer overflow, making \texttt{strncpy()} the ``safe version'' of \texttt{strcpy()<super>citation_needed<super>
\end{itemize}
\item Once again, \emph{you} must make sure:
\begin{itemize}
\item \texttt{s1} is large enough to contain \texttt{s2}
\item \texttt{n} takes null termination into accoun<super>citation_needed<super>  
\end{itemize}
\end{itemize}
\end{frame}

\begin{frame}[fragile=singleslide]{Yet More Functions!}
\begin{lstlisting}[style = C]
char *strcat(char *s1, const char *s2);
char *strncat(char *s1, const char *s2, size_t n);
size_t strlen(const char *s);
\end{lstlisting}
\begin{itemize}
\item \texttt{strcat()} usage is similar to \texttt{strcpy()}
\begin{itemize}
\item The null character terminating \texttt{s1} is overwritten with the first character of \texttt{s2<super>citation_needed<super>  
\end{itemize}
\item \texttt{strlen()} accepts a string and produces the number of characters in it, null character exclude<super>citation_needed<super>  
\end{itemize}
\end{frame}

\begin{frame}[fragile=singleslide]{My String's Better Than Your Strin<super>citation_needed<super><super>citation_needed<super> \texttt{strcmp}}
Ever needed to tell if two strings are the same string? Try \texttt{strcmp()} on for size! 
\begin{lstlisting}[style = C]
int strcmp(const char *s1, const char *s2);
int strncmp(const char *s1, const char *s2, size_t n);
\end{lstlisting}
\begin{itemize}
\item Inputs are the same as previously
\item Compares the characters in each string sequentially and returns:
\begin{itemize}
\item \texttt{0} if the strings are the sam<super>citation_needed<super>
\item A value less than 0 if s1 is ``less than'' s2
\item A value greater than 0 if s1 is ``greater than'' s<super>citation_needed<super>
\end{itemize}
\item In this context, strings are ordered by the ASCII values of their characters, using alphabetization rule<super>citation_needed<super>
\item This is known as \textit{Lexographical Orderin<super>citation_needed<super>}
\end{itemize}
\end{frame}

\begin{frame}{A Table with Equivalent Python Operations}
\center
\begin{tabular}{| l | c |}
\hline 
C function & Rough Python Equivalent \\ \hline
\texttt{strtol()} & \texttt{int(myString)} \\ \hline
\texttt{snprintf()} & \texttt{str(myInt)} \\ \hline
\texttt{strcpy()} & \texttt{cop<super>citation_needed<super>deepcopy(foo)} \\ \hline
\texttt{strcat()} & \texttt{foo + bar} \\ \hline
\texttt{strlen()} & \texttt{len(foo)} \\ \hline
\texttt{strcmp()} & [\texttt{==}, \texttt{\textless}, \texttt{\textgreater}, et<super>citation_needed<super>] \\ \hline
\texttt{strchr()} & \texttt{fo<super>citation_needed<super>index(bar)} \\ \hline
\texttt{strtok()} & \texttt{fo<super>citation_needed<super>split(bar)} \\ \hline
\end{tabular}
\end{frame}

% \begin{frame}{Surely This is Generalizabl<super>citation_needed<super>}
% Given that in C, data types as a concept are somewhat ephemeral, it should be the case that the above operations on \texttt{char*}s should be reasonably generalizabl<super>citation_needed<super>
% \begin{itemize}
% \item \texttt{strin<super>citation_needed<super>h} contains a set of memory operations which are highly similar to the operations on strings previously discusse<super>citation_needed<super> 
% \item Rather than accepting and returning \texttt{char*} pointers, these functions use \texttt{void*} pointer<super>citation_needed<super>  
% \end{itemize}
% \begin{tabular}{| l | c | c |}
% \hline 
% C function & C String Equivalent & Rough Python Equivalent \\ \hline
% \texttt{memcpy()} & \texttt{strcpy()} & \texttt{cop<super>citation_needed<super>deepcopy(foo)} \\ \hline
% \texttt{memcmp()} & \texttt{strcmp()} & [\texttt{==}, \texttt{\textless}, \texttt{\textgreater}, et<super>citation_needed<super>] \\ \hline
% \texttt{memchr()} & \texttt{strchr()} & \texttt{fo<super>citation_needed<super>index(bar)} \\ \hline
% \end{tabular}
% \end{frame}

\section[Formatting]{Advanced Formatting}
\begin{frame}{Formatting: What can't it do?}
Up to this point, we have only briefly touched on the advanced features of the string formatting tag \texttt{\%<super>citation_needed<super>  We will discuss the following features:
\begin{itemize}
\item Rounding of floating point numbers, and displaying a specified number of decimal place<super>citation_needed<super>
\item Aligning columns of numbers
\item Right and Left Justification of outputs
\item Exponential formats for floating point numbers
\item Fixed field widths for various data types
\end{itemize}
All of the format specifiers we are about to discuss are used in \texttt{printf()}, \texttt{scanf()} and their cousin<super>citation_needed<super>
\end{frame}

\begin{frame}{Format Specifiers for Integer Formats}
\center
\
\end{frame}

\begin{frame}{Printing Integers in Various Ways}
\center
\
\end{frame}

\begin{frame}{Printing Integers in Various Ways (con<super>citation_needed<super>)}
\center
\
\flushleft
Things of note:
\begin{itemize}
\item The literal \texttt{2000000000L} is cast as a \texttt{long int} with the suffix \texttt{L<super>citation_needed<super>
\item A negative number, interpreted as an unsigned integer, is very large! 
\begin{itemize}
\item This is because negative numbers are stored using \textit{Two's Complement}, which we'll be discussing soo<super>citation_needed<super>
\end{itemize}
\end{itemize}
\end{frame}

\begin{frame}{Printing Floating Point Numbers}
\center
\
\end{frame}

\begin{frame}{Example Floats}
\center
\
\end{frame}

\begin{frame}{Floating Some Idea<super>citation_needed<super><super>citation_needed<super>}
\begin{itemize}
\item \texttt{e} and \texttt{f} show six digits of precision by defaul<super>citation_needed<super>
\item \texttt{f} will always print at least one digit to the left of the decimal poin<super>citation_needed<super>  
\item \texttt{g} will select \texttt{e} if the exponent would be greater than the specified precision (default 6) or less than -4, and \texttt{f} otherwis<super>citation_needed<super> 
\item \texttt{g} does not print trailing zeroe<super>citation_needed<super>  
\item \texttt{e} and \texttt{g} display \emph{rounded} values, \texttt{f} displays \emph{truncated} value<super>citation_needed<super>  
\end{itemize}
Random Trivia: You can print the \texttt{\%} character with either \texttt{\%\%<super>citation_needed<super>  Note that \texttt{\textbackslash \%} doesn't wor<super>citation_needed<super>
\end{frame}

\begin{frame}{Working with Field Widths}
Ever been bummed out because getting C to print a table of values with properly aligned numbers is hard?  Fear no more! 
\begin{itemize}
\item Inserting an integer value between the \texttt{\%} character and the format specifier sets the \textbf{field width<super>citation_needed<super> 
\item Data will normally be \emph{right justified} within this fiel<super>citation_needed<super> 
\item Field widths may be used with all format specifier<super>citation_needed<super>  
\item If your field width is too narrow for your data, that data will ``overhang'' the specified width, rather than being truncate<super>citation_needed<super>  
\end{itemize}
\end{frame}

\begin{frame}{Field Widths}
\center
\
\end{frame}

\begin{frame}{Field Widths (con<super>citation_needed<super>)}
\center
\
\end{frame}

\begin{frame}{Field Widths with Precision}
We can specify various things for different data types using the \texttt<super>citation_needed<super>X} format ta<super>citation_needed<super>
\begin{itemize}
\item When applied to \texttt{int}s, leading zeros are included instead of spaces, to fill the field widt<super>citation_needed<super>
\item When applied to \texttt{float}s, this controls the number of decimal places that are displayed (<super>citation_needed<super><super>citation_needed<super>, the precision<super>citation_needed<super>
\item When applied to strings, this sets the number of characters to displa<super>citation_needed<super>  The rest of the string will be truncate<super>citation_needed<super>
\begin{itemize}
\item The \texttt{\%s} tag looks for a terminating null characte<super>citation_needed<super>
\item If you aren't careful, \texttt{\% s} can produce a segfault! 
\item Using \texttt{\% s} instead of \texttt{\% c} by mistake can also cause thi<super>citation_needed<super>
\item Specifying a field width is an effective countermeasure against a non-null-terminated strin<super>citation_needed<super>  
\end{itemize}
\end{itemize}
\end{frame}

\begin{frame}{Let's be Precise About This!}
\center
\
\end{frame}



\begin{frame}[fragile=singleslide]{Let's be Precise About This! (con<super>citation_needed<super>)}
\center
\
\flushleft
Field widths and precisions may be used together! 
\begin{lstlisting}[style = C]
printf("%<super>citation_needed<super>6f", 12<super>citation_needed<super>456789);
\end{lstlisting}
\hrule
\begin{lstlisting}[style=terminal]
__12<super>citation_needed<super>456
\end{lstlisting}
(The underscores above were added to visualize the spaces)
\end{frame}


\begin{frame}{Various Options}
In addition, there are a number of flags that may be included to modify print forma<super>citation_needed<super>
\center
\
\end{frame}

\section[File I/O]{Permanent Memory Interaction}
% \begin{frame}{Remember Kids, Don't Drink and Code!}
% \center
% \
% \end{frame}

\begin{frame}{File Operations}
The following five functions are the essential operations for reading from and writing to file<super>citation_needed<super>  
\begin{itemize}
\item \texttt{fopen()}, \texttt{fclose()}, \texttt{fscanf()}, \texttt{fread()}, \texttt{fwrite()}, \texttt{fprintf}, \texttt{feof()}
\end{itemize}
Using these files requires knowledge of the \texttt{FILE} structure, which is defined in \texttt{<stdi<super>citation_needed<super>h><super>citation_needed<super>  Think of it as a structure containing all the inforation relevant to a file that is currently being streame<super>citation_needed<super> 
\begin{itemize}
\item It will probably make more sense when we cover \texttt{struct} next wee<super>citation_needed<super>
\item This section assumes you at least vaguely remember how to do this in pytho<super>citation_needed<super>  
\end{itemize}
\end{frame}

\begin{frame}[fragile=singleslide]{\texttt{fopen()}}
Opening a file stream requires an invokation of \texttt{fopen()<super>citation_needed<super>
\begin{lstlisting}[style=C]
FILE *fopen(const char *filename, const char *mode);
\end{lstlisting}
\begin{itemize}
\item As you may expect, \texttt{filename} is a string, containing the name of the file to be opene<super>citation_needed<super>
\begin{itemize}
\item The file may be specified by either absolute or relative addressin<super>citation_needed<super>  
\end{itemize}
\item \texttt{mode} is a string specifying, among other things, whether the file will be opened in read or write mod<super>citation_needed<super>  
\item The output of this function is a \texttt{FILE} pointer to the data stream objec<super>citation_needed<super>  You don't have to worry about direct manipulation of this pointer, it is only interacted with via the file operation function<super>citation_needed<super>  
\end{itemize}
\end{frame}

\begin{frame}{Modus Operandi}
\center
\
\end{frame}

\begin{frame}[fragile=singleslide]{\texttt{fclose()}}
Just as a dynamically allocated pointer needs to be freed when it is no longer needed, a file stream must be closed when you're finished with i<super>citation_needed<super> 
\begin{lstlisting}[style=C]
int fclose(FILE *stream)
\end{lstlisting}
\begin{itemize}
\item This one's pretty straightforwar<super>citation_needed<super>  Takes the file stream as an argumen<super>citation_needed<super>
\item Returns 0 on success, EOF on failur<super>citation_needed<super>  
\end{itemize}
\end{frame}

\begin{frame}[fragile=singleslide]{\texttt{fscanf()}}
If you want to treat your file like you are reading information from \texttt{stdin}, consider using \texttt{fscanf()} and \texttt{fprintf()<super>citation_needed<super>
\begin{lstlisting}[style=C]
int fscanf(FILE *stream, const char *format,<super>citation_needed<super><super>citation_needed<super>)
\end{lstlisting}
\begin{itemize}
\item While the first argument is the file stream we are reading from, the rest of the arguments are used exactly like \texttt{scanf()<super>citation_needed<super>
\item The return value is the number of input items successfully matched and assigne<super>citation_needed<super> 
\begin{itemize}
\item <super>citation_needed<super><super>citation_needed<super>, the number of format specifiers in the format string that were successfu<super>citation_needed<super>  
\end{itemize}
\end{itemize}
\end{frame}

\begin{frame}[fragile=singleslide]{\texttt{fprintf()}}
\begin{lstlisting}[style=C]
int fprintf(FILE *stream, const char *format,<super>citation_needed<super><super>citation_needed<super>)
\end{lstlisting}
\begin{itemize}
\item Writes characters into a file in exactly the manner you would expec<super>citation_needed<super>
\item First argument is the filestrea<super>citation_needed<super>
\item Returns the number of characters written if successful, and a negative number upon failur<super>citation_needed<super>
\item nuff sai<super>citation_needed<super>
\end{itemize}
\end{frame}

\begin{frame}[fragile=singleslide]{\texttt{fread()}}
If repeatedly invoking \texttt{fscanf()} isn't your style, perhaps you'd prefer writing chunks of files directly into arrays?
\begin{lstlisting}[style=C]
size_t fread(void *ptr, size_t size, size_t nmemb, FILE *stream)
\end{lstlisting}
\begin{itemize}
\item \texttt{ptr} is a pointer to the memory block you're writing t<super>citation_needed<super>  It needs to be at least as big as \texttt{size*nmemb}
\item \texttt{size} is the size in bytes of each element to be rea<super>citation_needed<super>  
\begin{itemize}
\item So, using \texttt{sizeof()} would be a good idea!
\end{itemize}
\item \texttt{nmemb} is the number of elements to read, each the size of \texttt{size<super>citation_needed<super>
\item \texttt{stream} is, of course, the file strea<super>citation_needed<super>
\item \texttt{fread()} returns the total number of elements successfully rea<super>citation_needed<super>
\end{itemize}
\end{frame}


\begin{frame}[fragile=singleslide]{\texttt{fwrite()}}
And in revers<super>citation_needed<super><super>citation_needed<super>
\begin{lstlisting}[style=C]
size_t fwrite(const void *ptr, size_t size, size_t nmemb, FILE *stream)
\end{lstlisting}
\begin{itemize}
\item \texttt{ptr} is a pointer to the array of elements to be writte<super>citation_needed<super>
\item \texttt{size} is the size, in bytes, of each element of \texttt{ptr}
\item \texttt{nmemb} is the number of elements to be writte<super>citation_needed<super>
\item \texttt{stream} is, obviously, our old friend the file strea<super>citation_needed<super>  
\item \texttt{fwrite()} returns the total number of elements successfully written to the file strea<super>citation_needed<super>  
\end{itemize}
\end{frame}


\begin{frame}[fragile=singleslide]{\texttt{feof()}}
So all this reading from files is fine, but how do we know when we're finished? 
\begin{lstlisting}[style=C]
int feof(FILE *stream)
\end{lstlisting}
\begin{itemize}
\item \texttt{feof()} tests to see if the file stream has reached the end of the fil<super>citation_needed<super>
\item Input is the file stream (naturally)
\item Output is non-zero if the end of the file has been reached, zero otherwis<super>citation_needed<super>  
\end{itemize}
\end{frame}

\begin{frame}{An Example!}
\center
\
\end{frame}

\begin{frame}{An Example! (con<super>citation_needed<super>)}
\center
\
\emph{Cue Demo!}
\end{frame}

% \begin{frame}{The Last Slide Comic}
% \center
% \
% \end{frame}

\section[Acknowledge]{Acknowledge}
\begin{frame}{Acknowledge}
\center
\vspace{8em}
The contents of these slides were liberally borrowed (with permission) from slides from the Summer 2021 offering of 1XC3 (by D<super>citation_needed<super> Nicholas Moore<super>citation_needed<super>  
\end{frame}


\end{document}
