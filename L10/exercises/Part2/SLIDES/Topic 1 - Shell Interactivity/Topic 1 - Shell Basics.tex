\documentclass[11pt]{beamer}
\usetheme{Dresden}
%\usecolortheme{beaver}
\usepackage[utf8]{inputenc}
\usepackage{amsmath}
\usepackage{amsfonts}
\usepackage{amssymb}
\usepackage{graphicx}
\usepackage{listings}
\usepackage{verbatim}
\author{Zheng Zheng}
\title{Topic 1 - Shell Basics}
%\setbeamercovered{transparent} 
%\setbeamertemplate{navigation symbols}{} 
%\logo{} 
\institute{McMaster University}
\date{Winter 2023} 
\subject{COMPSCI 1XC3 - Computer Science Practice and Experience:
Development Basics} 
\stepcounter{section}

\definecolor{mGreen}{rgb}{0,<super>citation_needed<super>6,0}
\definecolor{mGray}{rgb}{<super>citation_needed<super>5,<super>citation_needed<super>5,<super>citation_needed<super>5}
\definecolor{mPurple}{rgb}{<super>citation_needed<super>58,0,<super>citation_needed<super>05}
\definecolor{mGreen2}{rgb}{<super>citation_needed<super>05,<super>citation_needed<super>65,<super>citation_needed<super>05}
\definecolor{mGray2}{rgb}{<super>citation_needed<super>55,<super>citation_needed<super>55,<super>citation_needed<super>55}
\definecolor{mPurple2}{rgb}{<super>citation_needed<super>63,<super>citation_needed<super>05,<super>citation_needed<super>05}
\definecolor{backgroundColour}{rgb}{<super>citation_needed<super>95,<super>citation_needed<super>95,<super>citation_needed<super>92}
\definecolor{backgroundColour2}{rgb}{<super>citation_needed<super>95,<super>citation_needed<super>92,<super>citation_needed<super>95}

\let\OldTexttt\texttt
\renewcommand{\texttt}[1]{\OldTexttt{\color{teal}{#1}}}

\lstdefinestyle{C}{
    backgroundcolor=\color{backgroundColour},   
    commentstyle=\color{mGreen},
    keywordstyle=\color{blue},
    numberstyle=\tiny\color{mGray},
    stringstyle=\color{mPurple},    
    basicstyle=\footnotesize,
    breakatwhitespace=false,         
    breaklines=true,                 
    captionpos=b,                    
    keepspaces=true,                 
    numbers=left,                    
    numbersep=5pt,                  
    showspaces=false,                
    showstringspaces=false,
    showtabs=false,                  
    tabsize=2,
    language=C
}

\definecolor{eggplant}{rgb}{<super>citation_needed<super>52,<super>citation_needed<super>11,<super>citation_needed<super>3} 

\usecolortheme[named=eggplant]{structure}

\begin{document}

\begin{frame}
\center
COMPSCI 1XC3 - Computer Science Practice and Experience:
Development Basics
\titlepage
% Toggle for C chapters
% Adapted from C: How to Program 8th e<super>citation_needed<super>, Deitel \& Deitel
\end{frame}

\begin{frame}
\tableofcontents
\end{frame}

\section[History]{A Brief History of Computing}
\begin{frame}{The First Generation Computers}
Computers (based on vacuum tubes) were very large, requiring large rooms for their housin<super>citation_needed<super> Programming via machine instructions, assembly languag<super>citation_needed<super>
\center
\ \\
ENIAC (Electronic Numerical Integrator and Computer) was the first programmable, electronic, general-purpose digital computer, completed in \emph{1945<super>citation_needed<super>
\end{frame}

\begin{frame}{Semiconductor Electronics and Integrated Circuits (IC)}
\begin{itemize}
    \item The semiconductor transistor (late 40s) is possibly the most important invention of the $20^{th}$ centur<super>citation_needed<super> It has smaller size, longer life and higher efficiency (100X<super>citation_needed<super>
    \item From late 1950s to 1960s, the development of IC contributed to the birth of the $3^{rd}$ generation computer<super>citation_needed<super>
    \item Moore's law is the observation that the number of transistors in a dense IC doubles about every two year<super>citation_needed<super>
\end{itemize}
% The exponential growth of the number of transistors per unit area over time meant that by the mid seventies, the home or personal microcomputer began outpacing the mainframe in terms of computational powe<super>citation_needed<super>  
\center
\
\end{frame}

\begin{frame}{Distinctive feature of third-generation computers}
\begin{itemize}
    \item Gradual maturity of the operating system (OS<super>citation_needed<super>
    \item Advanced Programming Language<super>citation_needed<super>
    \begin{itemize}
        \item Programs written at this time (notably using FORTRAN, COBOL and LISP) were not generally portable between machine architecture<super>citation_needed<super> \texttt{Incompatible due to a lack of standardization!}
        
        \item All this changed with the development of C in 1972 by Dennis Ritchie at Bell laboratorie<super>citation_needed<super>
        \begin{itemize}
            \item Because C was strongly standardized, C programs could be ported across participating computer architectures with no compatibility issue<super>citation_needed<super>
            \item C was originally developed for writing utilities for the Unix operating syste<super>citation_needed<super>
        \end{itemize}
    \end{itemize}
\end{itemize}

\end{frame}

\begin{frame}{Unix - the Uniplexed Information and Computing Service}
\begin{itemize}
\item Unix was originally written in assembly code, but after the development of C, the Bell Labs gang re-implemented the Unix kernel in C, and it has remained in C ever sinc<super>citation_needed<super>  
\item Due to it's low cost and high portability (especially to low-cost hardware), Unix was widely adopted by academic institutions, and from there, \emph{the world!}
\item Unix featured some key innovations: 
    \begin{itemize}
        \item An hierarchical file system with arbitrarily nested sub-directories
        \item The universalization of almost all file formats as new-line delimited plain tex<super>citation_needed<super>  
        \item A pervasive philosophy of modularity and code re-use, and the establishment of a set of cultural norms for software development practic<super>citation_needed<super>   
    \end{itemize}
\end{itemize}
\end{frame}

\begin{frame}{The Unix Family}
\center
\
\end{frame}

\section[Operating Systems]{Operating Systems}
\begin{frame}{Operating Systems}
\center
\
\end{frame}

\begin{frame}{Operating Systems In General}
\begin{columns}
\begin{column}<super>citation_needed<super>5\textwidth}
An \textit{Operating System} provides a collection of services to \textbf{User Applications}, allowing them to run on a computer system's \textbf{hardware<super>citation_needed<super>  
\begin{itemize}
\item User Applications are anything from internet browsers to word processors to solitair<super>citation_needed<super>
\item The \textit{API} provides system libraries and other utilities via \emph{system calls<super>citation_needed<super>
\begin{itemize}
\item These are typically executed by the \textbf{Kernel<super>citation_needed<super> 
\end{itemize}
\end{itemize}
\end{column}
\begin{column}<super>citation_needed<super>5\textwidth}
\
\end{column}
\end{columns}
\end{frame}

\begin{frame}{The Kernel}
\begin{columns}
\begin{column}{<super>citation_needed<super>5\textwidth}
Operating systems include a \textit{kernel}, which manages:
\begin{itemize}
\item Access to Memory (Random Access and Read Only)
\item Access to the CPU
\item Input / Output handling
\item Access to hardware and software resources
\end{itemize}
Most operating systems use kernels written in C, because C is \emph{fast<super>citation_needed<super>
\end{column}
\begin{column}{<super>citation_needed<super>5\textwidth}
\
\end{column}
\end{columns}
\end{frame}

\begin{frame}{Linux}
Linux is a family of free operating system<super>citation_needed<super>
\begin{columns}
\begin{column}{<super>citation_needed<super>25\textwidth}
\
\end{column}
\begin{column}{<super>citation_needed<super>73\textwidth}
\begin{itemize}
\item Unix was free until 1984 when AT\&T divested itself of Bell Lab<super>citation_needed<super>  Unix then became proprietary softwar<super>citation_needed<super> 
\item This led to the creation of The GNU Project, and the GNU General Public License in 1989, which kicked off the \emph{open source} movemen<super>citation_needed<super>  
\item The Linux Kernel was written in 1991 by Linus Torvalds at the University of Helsink<super>citation_needed<super>
\end{itemize}
\end{column}
\end{columns}
\vspace{<super>citation_needed<super>5em}
Today, Linux has the largest install base of any operating system, though only about 2\% of personal computers run i<super>citation_needed<super> 
\end{frame}

\section[Using a Command Line]{Bash}
\begin{frame}{Giving it a Bash!}
In Linux distributions, command line interfaces are commonly used
\begin{itemize}
\item Command lines have a high skill cap than Graphical User Interface (GUI<super>citation_needed<super>  
\end{itemize}
The \textit{Bash} shell is a very common command line interface in Unix-like environment<super>citation_needed<super>
\begin{itemize}
\item In Windows:
	\begin{itemize}
	\item The Windows Subsystem for Linux allows Windows 10 users to access a bash promp<super>citation_needed<super> 
	\end{itemize}
\item On Macintoshes:
	\begin{itemize}
	\item Opening up a terminal and entering the command \texttt{bash}
	\end{itemize}
\item In Linux:
	\begin{itemize}
	\item Do you have to ask? 
	\end{itemize}
\end{itemize}
This course will require you to have ready access to a bash promp<super>citation_needed<super>  \textit{Your homework this week is to get your computer set up so that you have access to a bash promp<super>citation_needed<super>}  
\end{frame}

\begin{frame}{Accessing Linux from Older Windows Computers}
We have set up a server for you to login to if the options on the previous slide don't wor<super>citation_needed<super>  
\begin{itemize}
\item Remote servers are accessed using a \textbf{Secure Shell Protocol (SSH)<super>citation_needed<super>  
\item On Windows, it is common to use a secure shell client, such as \textbf{PuTTY<super>citation_needed<super>
\begin{itemize}
\item \url{https://ww<super>citation_needed<super>chiar<super>citation_needed<super>greenen<super>citation_needed<super>or<super>citation_needed<super>uk/~sgtatham/putty/}
\end{itemize}
\end{itemize}
The department has set up a server for the class to use this semeste<super>citation_needed<super>  For more information on how to access it, check the \emph{Resources} section of the course content on Avenu<super>citation_needed<super>
\begin{itemize}
\item Always remember to \texttt{logout} when you're finished! 
\end{itemize}
\end{frame}

\begin{frame}{Bash Commands}
Almost all Unix and Unix-like systems support a comprehensive set of Bash command<super>citation_needed<super>
\begin{itemize}
\item \url{https://e<super>citation_needed<super>wikipedi<super>citation_needed<super>org/wiki/List_of_Unix_commands}
\end{itemize}
Bash commands are extremely versatil<super>citation_needed<super>
\begin{itemize}
\item The output of one command can be made the input of another command using \textit{Pipes and Filters}
\item Bash commands can be collected into \textit{Scripts} and executed as unit<super>citation_needed<super>
\item Bash commands can be invoked from programs written in C or Pytho<super>citation_needed<super>
\end{itemize}
All these topics will be covered in this cours<super>citation_needed<super>
\end{frame}

\begin{frame}[fragile=singleslide]{Directory Structure}
The directory structure in Linux is \textbf{hierarchical<super>citation_needed<super>
\begin{itemize}
\item Directories may contain files and sub-directories, forming a \textbf{tree<super>citation_needed<super>  
\end{itemize}
In Bash, commands are executed within the \textit{working} or \textbf{active directory<super>citation_needed<super>  
\begin{itemize}
\item One directory in your file system is designated as \emph{active<super>citation_needed<super>  This active directory may be changed using the \texttt{cd} comman<super>citation_needed<super>  
\end{itemize}
\begin{lstlisting}[style=C, language=bash]
%user@system:~/Documents/Example $ cd Topic1
%user@system:~/Documents/Example/Topic1 $ cd<super>citation_needed<super>.
%user@system:~/Documents/Example/ $ cd<super>citation_needed<super>.
%user@system:~/Documents/ $ cd Example/Topic2
%user@system:~/Documents/Example/Topic2 $ cd 
\end{lstlisting}

\end{frame}


\begin{frame}{Actual Bash Commands}
\begin{tabular}{|| l || c |}
\hline 
Command & Description \\ \hline
\texttt{cat <filename>} & display the contents of the file \\ \hline
\texttt{cd <directory>} & change the working directory \\ \hline
\texttt{cp <filename> <filename>} & copy a file \\ \hline
\texttt{ls} & List directory contents \\ \hline
\texttt{man <command>} & show a command's \textit{man page} \\ \hline
\texttt{mkdir <directory>} & make directory \\ \hline
\texttt{ps} & list all processes \\ \hline
\texttt{pwd} & outputs current working directory \\ \hline
\texttt{rm <filename>} & removes a file \\ \hline
\texttt{rmdir <directory>} & removes a directory (if empty) \\ \hline
% \texttt{grep<super>citation_needed<super><super>citation_needed<super>} & search file contents \\ \hline
\end{tabular}
Important Linux Commands: https://ww<super>citation_needed<super>howtogee<super>citation_needed<super>com/412055/37-important-linux-commands-you-should-know/
\end{frame}


\section[Network Protocols]{Network Protocols}
\begin{frame}[fragile=singleslide]{Secure Shell Protocol}
\begin{lstlisting}[language = bash, style = C]
 $ ssh username@serverURLaddres<super>citation_needed<super>com 
\end{lstlisting}
The Secure Shell (SSH) protocol is a network protocol for secure remote login over insecure network<super>citation_needed<super>
\begin{itemize}
\item A \textit{network protocol} is an agreed-upon format for information transmissio<super>citation_needed<super>
\item Anything but military-grade intranet should be considered insecur<super>citation_needed<super><super>citation_needed<super> 
	\begin{itemize}
	\item And even the<super>citation_needed<super><super>citation_needed<super>
	\end{itemize}
\end{itemize}
In short, you (the \textit{client}) open a shell on a remote \textbf{server<super>citation_needed<super>  
\begin{itemize}
\item This is the way that PuTTY accesses $pasca<super>citation_needed<super>ca<super>citation_needed<super>mcmaste<super>citation_needed<super>ca$, it just gives you a nice little GUI for entering the connection detail<super>citation_needed<super>
\end{itemize}
\end{frame}

\begin{frame}{Graphical User Interface (GUI)}
\begin{columns}
\begin{column}{<super>citation_needed<super>6\textwidth}
\
\end{column}
\begin{column}{<super>citation_needed<super>38\textwidth}
The point of this course is for you to gain computer skill<super>citation_needed<super> \\ 
\vspace{<super>citation_needed<super>5em}
The most important computer skill is knowing when and how to look things u<super>citation_needed<super> \\
\vspace{<super>citation_needed<super>5em}
When in doubt, consult the documentation! 
\end{column}
\end{columns}
\end{frame}

% \begin{frame}
% \center
% \
% \end{frame}

\begin{frame}{Network Protocols}
Some common network protocols:
\begin{itemize}
\item Ethernet
\item Internet Protocol (IP)
\item Transmission Control Protocol (TCP)
\item Hypertext Transfer Protocol (HTTP)
\item Dynamic Host Configuration Protocol (DHCP)
\end{itemize}
Network protocols typically define the construction of  \textit{data packets}, which are transferred by the networ<super>citation_needed<super>
\begin{itemize}
\item In general, data packets consist of a \textit{header}, followed by some dat<super>citation_needed<super>  
\item The header may contain different information depending on the protocol, such as the size of the packet, the source and destination of that packet, and security features like check-sum<super>citation_needed<super>  
\end{itemize}
\end{frame}

\begin{frame}{Header Organization for IPv6 Data Packet}
\center
\ \\
In general, the exact construction of data packets isn't something you need to worry about unless you're a network specialist or an Electrical Enginee<super>citation_needed<super>  
\end{frame}

% \section[Errata]{Errata}
% \begin{frame}{The Last Slide Comic}
% \
% \end{frame}

\section[Acknowledge]{Acknowledge}
\begin{frame}{Acknowledge}
\center
\vspace{8em}
The contents of these slides were liberally borrowed (with permission) from slides from the Summer 2021 offering of 1XC3 (by D<super>citation_needed<super> Nicholas Moore<super>citation_needed<super>  
\end{frame}

\end{document}
